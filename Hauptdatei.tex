\newcommand{\pdftitel}{PDFTITEL}
\newcommand{\autor}{Autor}
\newcommand{\arbeit}{ART DER ARBEIT}
%
% Nahezu alle Einstellungen koennen hier getaetigt werden
%

\documentclass[%
	pdftex,
	oneside,			% Einseitiger Druck.
	11pt,				% Schriftgroesse
	parskip=half,		% Halbe Zeile Abstand zwischen Absätzen.
	%headsepline,		% Linie nach Kopfzeile.
	%footsepline,		% Linie vor Fusszeile.
	abstracton,	   	 	% Abstract Überschriften
	ngerman,			% Translator
	pointlessnumbers	% Keine Punkte nach Nummerierung
]{scrreprt}

%Seitengroesse
%\usepackage{fullpage}

%Zeilenumbruch und mehr
\usepackage[activate]{microtype}

%Seitenzahl am Blatt Anfang
\usepackage[manualmark]{scrpage2}
\clearscrheadfoot
\chead[\pagemark]{\pagemark}
\pagestyle{scrheadings}

%Ränder
\usepackage{geometry}
\geometry{left=35mm, right=20mm, top=25mm, bottom=20mm}

% Zeichencodierung
\usepackage[utf8]{inputenc}
\usepackage[T1]{fontenc}

% Zeilenabstand
\usepackage[onehalfspacing]{setspace}

% Index-Erstellung
\usepackage{makeidx}

% Lokalisierung (neue deutsche Rechtschreibung)
\usepackage[ngerman]{babel}

% Anführungszeichen 
\usepackage[babel,german=quotes]{csquotes}
%\usepackage[style=swiss]{csquotes}


% Spezielle Tabellenform fuer Deckblatt
\usepackage{longtable}
\setlength{\tabcolsep}{10pt} %Abstand zwischen Spalten
\renewcommand{\arraystretch}{1.5} %Zeilenabstand

% Grafiken
\usepackage{graphicx}
\usepackage{subfigure}

% Mathematische Textsaetze
\usepackage{amsmath}
\usepackage{amssymb}

% Pakete um Textteile drehen zu können, oder eine Seite Querformat anzeigen kann.
%\usepackage{rotating}
\usepackage{lscape}

% Farben
\usepackage{color}
\definecolor{LinkColor}{rgb}{0,0,0.2}
\definecolor{ListingBackground}{rgb}{0.92,0.92,0.92}


% Titel, Autor und Datum
\title{\titel}
\author{\autor}
\date{\datum}

% PDF Einstellungen
\usepackage[%
	pdftitle={\pdftitel},
	pdfauthor={\autor},
	pdfsubject={\arbeit},
	pdfcreator={pdflatex, LaTeX with KOMA-Script},
	pdfpagemode=UseOutlines, % Beim Oeffnen Inhaltsverzeichnis anzeigen
	pdfdisplaydoctitle=true, % Dokumenttitel statt Dateiname anzeigen.
	pdflang=de % Sprache des Dokuments.
]{hyperref} 

% (Farb-)einstellungen für die Links im PDF
\hypersetup{%
	colorlinks=false, % Aktivieren von farbigen Links im Dokument
	linkcolor=LinkColor, % Farbe festlegen
	citecolor=LinkColor,
	filecolor=LinkColor,
	menucolor=LinkColor,
	urlcolor=LinkColor,
	bookmarksnumbered=true % Überschriftsnummerierung im PDF Inhalt anzeigen.
}

\usepackage[scaled]{uarial}
\renewcommand*\familydefault{\sfdefault} %% Only if the base font of the document is to be sans serif
\usepackage[T1]{fontenc}

% Verschiedene Schriftarten
%\usepackage{goudysans}
%\usepackage{lmodern}
%\usepackage{libertine}
%\usepackage{palatino}
 

% Hurenkinder und Schusterjungen verhindern
% http://projekte.dante.de/DanteFAQ/Silbentrennung
\clubpenalty=10000
\widowpenalty=10000
\displaywidowpenalty=10000

% Quellcode
\usepackage{listings}
%\lstloadlanguages{Java}
\lstset{%
	language=PSTricks,		 	 % Sprache des Quellcodes
	%numbers=left,           % Zelennummern links
	stepnumber=1,            % Jede Zeile nummerieren.
	numbersep=5pt,           % 5pt Abstand zum Quellcode
	numberstyle=\tiny,       % Zeichengrösse 'tiny' für die Nummern.
	breaklines=true,         % Zeilen umbrechen wenn notwendig.
	breakautoindent=true,    % Nach dem Zeilenumbruch Zeile einrücken.
	postbreak=\space,        % Bei Leerzeichen umbrechen.
	tabsize=2,               % Tabulatorgrösse 2
	basicstyle=\ttfamily\footnotesize, % Nichtproportionale Schrift, klein für den Quellcode
	showspaces=false,        % Leerzeichen nicht anzeigen.
	showstringspaces=false,  % Leerzeichen auch in Strings ('') nicht anzeigen.
	extendedchars=true,      % Alle Zeichen vom Latin1 Zeichensatz anzeigen.
	captionpos=b,            % sets the caption-position to bottom
	backgroundcolor=\color{ListingBackground} % Hintergrundfarbe des Quellcodes setzen.
}


%Akronyme
\usepackage[printonlyused,footnote]{acronym}

% Fussnoten
\usepackage[perpage, hang, multiple, stable]{footmisc}

%Bildpfad
\graphicspath{{images/}}

%nur ein latex-Durchlauf für die Aktualisierung von Verzeichnissen nötig
\usepackage{bookmark}

%Gleitumgebungen (Bilder, Tabellen, usw\ldots) lassen sich mit H an genau der
% definierten Stelle platzieren
%\usepackage{float}

% für die vertikale Platzierung von Text in Tabellen
\usepackage{array}

% für die Darstellung des Euro-Symbols
\usepackage[right]{eurosym}

% für textumflossene Grafiken
\usepackage{wrapfig}

% eine Kommentarumgebung "k" (Handhabe mit \begin{k}<Kommentartext>\end{k},
% Kommentare werden rot gedruckt). Wird \% vor excludecomment{k} entfernt,
% werden keine Kommentare mehr gedruckt.
\usepackage{comment}
\specialcomment{k}{\begingroup\color{red}}{\endgroup}
%\excludecomment{k}

%pdf anhängen
\usepackage{pdfpages}

%Tabellenzellen verbinden
%\usepackage{multirowbt}
\usepackage{multirow} 
%\usepackage{packages/multirowbt}
\usepackage{booktabs}

%Tabellenhintergrundfarbe anpassen
\usepackage{color}

% Define user colors using the RGB model
\definecolor{dunkelgrau}{rgb}{0.8,0.8,0.8}
\definecolor{hellgrau}{rgb}{0.95,0.95,0.95}

\usepackage{colortbl}

% Innerhalb Tabelle Spaltenbreite begrenzen und zentrieren
\usepackage{tabularx}
\newcolumntype{L}[1]{>{\raggedright\arraybackslash}p{#1}} % linksbündig mit Breitenangabe; Aufruf mit: L{Xcm}
\newcolumntype{C}[1]{>{\centering\arraybackslash}p{#1}} % zentriert mit Breitenangabe
\newcolumntype{R}[1]{>{\raggedleft\arraybackslash}p{#1}} % rechtsbündig mit Breitenangabe

%für mehr als 30 Bilder
%\usepackage[section] {placeins}
\usepackage{morefloats}


% Ab jetzt können auch Umlaute verwendet werden

\newcommand{\titel}{TITEL}
\newcommand{\martrikelnr}{Matrikelnr}
\newcommand{\kurs}{Kurs}
\newcommand{\datumAbgabe}{01.09.2014}
\newcommand{\firma}{Firma}
\newcommand{\firmenort}{Firmenort}
\newcommand{\abgabeort}{Stuttgart}
\newcommand{\abschluss}{Bachelor of Science}
\newcommand{\studiengang}{Wirtschaftsinformatik}
\newcommand{\dhbw}{Stuttgart}
\newcommand{\betreuertitel}{Betreuertitel}
\newcommand{\betreuerfirma}{Betreuername}
\newcommand{\funktion}{Betreuerfunktion}
\newcommand{\betreuer}{DH-Betreuer}
\newcommand{\zeitraum}{12 Wochen}
\newcommand{\arbeitsart}{1.Projektarbeit}
\newcommand{\dnd}{D\& D}

\begin{document}

	% Deckblatt
	\begin{spacing}{1}
	\setlength{\hoffset}{-10mm}
		\begin{titlepage}
	\begin{longtable}{p{.4\textwidth} p{.4\textwidth}}
	\end{longtable}
	%\enlargethispage{20mm}
	\begin{center}
	\begin{doublespace}
	  \vspace*{12mm}	{\LARGE\bf \titel }\\
	  \vspace*{12mm}	{\large\bf \arbeit}\\
	\end{doublespace}  
	  \vspace*{12mm}	vorgelegt am \datumAbgabe\\
	  \vspace*{12mm}	Fakultät Wirtschaft\\
	  \vspace*{3mm} 	Studiengang \studiengang\\
	 \vspace*{3mm} 	{\kurs}\\
	  \vspace*{12mm}	von\\
	  \vspace*{3mm} 	{\large\bf \autor}\\
	  
	  
	\end{center}
	\vfill
	\begin{spacing}{1.2}
	\begin{tabbing}
		mmmmmmmmmmmmmmmmmmmmmmmmmm     \= \kill
%		\textbf{Bearbeitungszeitraum}  \>  \zeitraum\\
		\textbf{Ausbildungsstätte:}	   \> \textbf{DHBW Stuttgart:} \\
		\firma						   \>  \\
		\betreuertitel				   \>  \\
		\betreuerfirma				   \>  \betreuer\\
		\funktion				   \>  \\
	\end{tabbing}
	\end{spacing}
	
	%Vertraulichkeitsvermerk bei Bedarf auskommentieren
	\begin{center}
	\textbf{Vertraulich}\\
	Der Inhalt der Arbeit darf Dritten ohne Genehmigung der Ausbildungsstätte nicht zugänglich gemacht werden.
	\end{center}
	
\end{titlepage}

	\end{spacing}
	\newpage

	\renewcommand{\thepage}{\Roman{page}}
	\setcounter{page}{1}

	\pagestyle{plain}

	% Inhaltsverzeichnis
	\begin{spacing}{1.1}
		\setcounter{tocdepth}{1}
		%für die Anzeige von Unterkapiteln im Inhaltsverzeichnis
		\setcounter{tocdepth}{2}
		\tableofcontents
	\end{spacing}
	\newpage
	
	% Abkürzungsverzeichnis
	\cleardoublepage
	\phantomsection \label{listofacs}
	\addcontentsline{toc}{chapter}{Abkürzungsverzeichnis}
	\chapter*{Abkürzungsverzeichnis}
%nur verwendete Akronyme werden letztlich im Dokument angezeigt
\begin{acronym}[YTMMM]
\setlength{\itemsep}{-\parsep}

\acro{Apps}{Applikationen}
\acro{CSS}{Cascasding Style Sheets}
\acro{HTML5}{Hypertext Markup Language 5}
\acro{XML}{Extensible Markup Language}
\acro{DOM}{Document Object Model}
\acro{REST}{Representational State Transfer}

\end{acronym}

	\newpage
	
	% Abbildungsverzeichnis
	\cleardoublepage
	\phantomsection \label{listoffig}
	\addcontentsline{toc}{chapter}{Abbildungsverzeichnis}
	\listoffigures
	\newpage	
	
	\renewcommand{\thepage}{\arabic{page}}
	\setcounter{page}{1}
	
	% Hier den Inhalt einfügen
	%\input{kapitel/kapitel01}


	% Literaturverzeichnis
	\cleardoublepage
	\phantomsection \label{listoflit}
	\addcontentsline{toc}{chapter}{Literaturverzeichnis}
	\begin{thebibliography}{\hspace{25mm}}


\bibitem[Backbone]{bip:BackboneEntwickeln}
	\textsc{Osmani, Addy (2013: Developing Backbone.js Applications}\\
	\textbf{First Revision, O'Reilly, Sebastopol}\\
	Einsichtnahme: 11.08.2014

	
\bibitem[Businessweek Geschichte Smartphones]{bip:BW}
	\url{http://www.businessweek.com/articles/2012-06-29/before-iphone-and-android-came-simon-the-		firstsmartphone}\\
	Einsichtnahme	25.08.2014
	
	
\end{thebibliography}

	\newpage
	
	% Erklärung
	\thispagestyle{empty}

\section*{Erklärung}
\vspace*{2em}

Ich erkläre hiermit ehrenwörtlich: \\
\begin{enumerate}
\item dass ich meine {\arbeitsart} mit dem Thema
{\itshape \titel } ohne fremde Hilfe angefertigt habe;
\item dass ich die Übernahme wörtlicher Zitate aus der Literatur sowie die Verwendung der Gedanken
anderer Autoren an den entsprechenden Stellen innerhalb der Arbeit gekennzeichnet habe;
\item dass ich meine {\arbeitsart} bei keiner anderen Prüfung vorgelegt habe;
\item dass die eingereichte elektronische Fassung exakt mit der eingereichten schriftlichen Fassung
übereinstimmt.
\end{enumerate}

Ich bin mir bewusst, dass eine falsche Erklärung rechtliche Folgen haben wird.

\vspace{3em}

\abgabeort, \datumAbgabe
\vspace{4em}

\autor

	\newpage

\end{document}
